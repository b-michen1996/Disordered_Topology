	\documentclass[english]{scrartcl}
	\usepackage[utf8]{inputenc}  
	\usepackage[T1]{fontenc}  
	
\usepackage{amsmath}
\usepackage{amssymb}
\usepackage{bbm}
\usepackage{bm}
\usepackage{tikz-cd} 
\usepackage[colorlinks,bookmarks=true,citecolor=blue,linkcolor=red,urlcolor=blue]{hyperref}

\usepackage{graphicx}  

\usepackage{caption}  
%\usepackage{subcaption}  
%\usepackage{float}   
\usepackage{subfig}
\usepackage{floatrow}
\usepackage{wrapfig}
\usepackage{stackengine}
\usepackage{tensor}

\usepackage{braket}
\usepackage{comment}

\usepackage[acronym]{glossaries}

\title{Backfolding bands}

\author{Benjamin Michen}
\date{\today}

\newcommand{\mc}{\mathcal}
\newcommand{\mb}{\mathbb}
\newcommand{\tb}{\textbf}
\newcommand{\ti}{{\hat T_I}}
\newcommand{\eq}[1]{Eq.~(\ref{#1})}



\makenoidxglossaries

\newacronym{bch}{BCH}{Baker-Campbell-Hausdorff}
\newacronym{1d}{1D}{one-dimensional}
\newacronym{2d}{2D}{two-dimensional}
\newacronym{fbz}{FBZ}{first Brillouin zone}
\newacronym{obc}{OBC}{open boundary conditions}

\glsdisablehyper


\begin{document}
\maketitle
{
\hypersetup{linkcolor=black}
\tableofcontents
}

\newpage

\section{Backfolding of a single band in one dimension}
We start from a \gls{1d} Hamiltonian with a single band:
\begin{align}
H_\mathrm{1d} = \sum_k \epsilon(k) c^\dagger_k c_k. \label{Eqn:H1d_generic}
\end{align}
We can fold the bands back into the first Half of the \gls{fbz} by artificially enlarging the unit cell to two sites: we call the even sites sublattice a and the odd sites sublattice b. This leads to a two-banded Hamiltonian with momentum $\tilde k$ discretized as $\Delta_{\tilde k} = \frac{4 \pi}{N}$ instead of $\Delta_{k} = \frac{2 \pi}{N}$ as before. This Hamiltonian is readily found by decomposing the creation operator in momentum space (we assume an even total number of sites $N$):

\begin{align}
c_k^\dagger =& \frac{1}{\sqrt{N}} \sum_{j = 0}^{N-1} \left [ e^{ik j} c_j^\dagger \right] = \frac{1}{\sqrt{2}} \frac{1}{\sqrt{N/2}} \sum_{l = 0}^{N/2 -1} \left [e^{ik 2l} c_{2l}^\dagger + e^{ik (2l + 1)} c_{2l + 1}^\dagger \right] \nonumber \\
=& \begin{cases}
&\frac{1}{\sqrt{2}}   \left [c_{2k, a}^\dagger  + e^{i k} c_{2k, b}^\dagger \right]  \text{if $|k| \leq \frac{\pi}{2}$} \\ \\
&\frac{1}{\sqrt{2}}   \left [c_{2k \mp 2 \pi, a}^\dagger  + e^{i k} c_{2k \mp 2 \pi, b}^\dagger \right]  \text{if $\pm|k| \geq \frac{\pi}{2}$}. 
\end{cases} \label{Eqn:c_k_backfolding}
\end{align}
Note that the two case of the above equation coincide at $k  =\pm \pi$!

The momentum operators of the two sublattices lattice are given by
\begin{align*}
c_{\tilde k, \gamma}^\dagger = \frac{1}{\sqrt{N/2}} \sum_{l = 0}^{N/2 - 1} e^{i \tilde k  l} c_l^\dagger.
\end{align*}

We can now express $H_\mathrm{1d}$ in terms of the new momentum operators by splitting the sum from \eq{Eqn:H1d_generic} into two parts and using \eq{Eqn:c_k_backfolding}:

\begin{align*}
H_\mathrm{1d} =& \sum_{- \pi \leq k < \pi} \epsilon(k) c^\dagger_k c_k \\
 =&  \sum_{- \pi / 2 \leq k < \pi/2} \frac{1}{2} \epsilon(k) \left[ c^\dagger_{2k, a} + e^{ik} c^\dagger_{2k, b} \right]  \left[ c_{2k, a} + e^{-ik} c_{2k, b} \right] \\
&+ \sum_{\pi / 2 \leq k < \pi} \frac{1}{2} \epsilon(k) \left[ c^\dagger_{2k - 2 \pi, a} + e^{ik} c^\dagger_{2k - 2 \pi, b} \right]  \left[ c_{2k - 2 \pi, a} + e^{-ik} c_{2k - 2 \pi, b} \right] \\
&+ \sum_{- \pi \leq k < \pi/2} \frac{1}{2} \epsilon(k) \left[ c^\dagger_{2k + 2 \pi, a} + e^{ik} c^\dagger_{2k + 2 \pi, b} \right]  \left[ c_{2k + 2 \pi, a} + e^{-ik} c_{2k + 2 \pi, b} \right] \\
 =&  \sum_{-\pi \leq \tilde k < \pi} \frac{1}{2}  \epsilon(\tilde k / 2) \left[ c^\dagger_{\tilde k, a} + e^{i \tilde k / 2} c^\dagger_{\tilde k, b} \right]  \left[ c_{\tilde k, a} + e^{-i \tilde k / 2} c_{\tilde k, b} \right] \\
&+ \sum_{- \pi \leq \tilde k < 0} \frac{1}{2} \epsilon(\tilde k / 2 + \pi) \left[ c^\dagger_{\tilde k, a} + e^{i (\tilde k / 2 + \pi)} c^\dagger_{\tilde k \pi, b} \right]  \left[ c_{\tilde k \pi, a} + e^{-i(\tilde k / 2 + \pi)} c_{\tilde k, b} \right] \\
&+ \sum_{0 \leq \tilde k < \pi} \frac{1}{2} \epsilon(\tilde k / 2 - \pi) \left[ c^\dagger_{\tilde k, a} + e^{i(\tilde k / 2 - \pi)} c^\dagger_{\tilde k, b} \right]  \left[ c_{\tilde k, a} + e^{-i(\tilde k / 2 - \pi)} c_{\tilde k, b} \right] 
\end{align*}
We use that $\epsilon(k)$ is $2\pi$-periodic and summarize this as 

\begin{align*}
H_\mathrm{1d} =& \frac{1}{2} \sum_{-\pi \leq \tilde k < \pi} \left \{ \left[\epsilon(\tilde k / 2) + \epsilon(\tilde k / 2 + \pi) \right]  \left[ c^\dagger_{\tilde k, a}c_{\tilde k, a} + c^\dagger_{\tilde k, b}c_{\tilde k, b} \right] \right. \\
& \left . + \left[\epsilon(\tilde k / 2) - \epsilon(\tilde k / 2 + \pi) \right] \left[e^{-i\tilde k / 2} c^\dagger_{\tilde k, a}c_{\tilde k, b} + e^{i\tilde k / 2} c^\dagger_{\tilde k, b}c_{\tilde k, a} \right] \right \} \\
=& \sum_{-\pi \leq \tilde k < \pi} \left(c^\dagger_{\tilde k, a}, c^\dagger_{\tilde k, b}\right) 
\left[ d_0(\tilde k) \sigma^0 + \bm d(\tilde k) \cdot \bm \sigma \right] 
\begin{pmatrix}
c_{\tilde k, a} \\
c_{\tilde k, b}
\end{pmatrix},
\end{align*}
where $d_0(\tilde k) = [\epsilon(\tilde k / 2) + \epsilon(\tilde k / 2 + \pi)] / 2$, $d_x(\tilde k) = \cos(\tilde k /2)[\epsilon(\tilde k / 2) - \epsilon(\tilde k / 2 + \pi)] / 2$, $d_y(\tilde k) = \sin(\tilde k /2)[\epsilon(\tilde k / 2) - \epsilon(\tilde k / 2 + \pi)] / 2$, and $d_z(\tilde k) = 0$.

\section{Backfolding of 2d Hamiltonians}
\subsection{Single band}
Consider a \gls{2d} single band Hamiltonian
\begin{align}
H_\text{2d, single band} = \sum_{k_x, k_y} \epsilon(k_x, k_y) c^\dagger_{k_x, k_y} c_{k_x, k_y}. \label{Eqn:H2d_single_band}
\end{align}
We consider the odd and even sites along $y$ direction as two sublattices and proceed as before. Everything works out analogously because $k_x$ is completely unaffected by what we do in the $y$ direction, leading to 

\begin{align*}
H_\text{2d, single band} =& \sum_{k_x, \tilde k_y} \left(c^\dagger_{k_x, \tilde k_y, a}, c^\dagger_{k_x, \tilde k_y, b}\right) 
\left[ d_0(k_x, \tilde k_y) \sigma^0 + \bm d(k_x, \tilde k_y) \cdot \bm \sigma \right] 
\begin{pmatrix}
c_{k_x, \tilde k_y, a} \\
c_{k_x, \tilde k_y, b}
\end{pmatrix},
\end{align*}
where $d_0(k_x, \tilde k_y) = [\epsilon(k_x, \tilde k_y / 2) + \epsilon(k_x, \tilde k_y / 2 + \pi)] / 2$, $d_x(k_x, \tilde k_y) = \cos(\tilde k_y /2)[\epsilon(k_x, \tilde k_y / 2) - \epsilon(k_x, \tilde k_y / 2 + \pi)] / 2$, $d_y(k_x, \tilde k_y) = \sin(\tilde k_y /2)[\epsilon(k_x, \tilde k_y / 2) - \epsilon(k_x,\tilde k_y / 2 + \pi)] / 2$, and $d_z(k_x, \tilde k_y) = 0$. 

We see that $H(k_x, \tilde k_y)$ cannot contain a $\sigma^z$ part and therefore has zero Chern number for all bands. Furthermore, it does not even commute with any Hamiltonian that contains a $\sigma^z$!

\subsection{Two bands}
We consider now a two-band Hamiltonian 
\begin{align*}
H_\text{2d, two band} =& \sum_{k_x, k_y} \left(c^\dagger_{k_x, k_y, \alpha}, c^\dagger_{k_x, k_y, \beta}\right) 
\left[ d_0(k_x, k_y) \sigma^0 + \bm d(k_x, k_y) \cdot \bm \sigma \right] 
\begin{pmatrix}
c_{k_x, k_y, \alpha} \\
c_{k_x, k_y, \beta}
\end{pmatrix},
\end{align*}
We fold this back along $k_y$ to obtain a four-band model by a dividing the $\alpha$ sublattice  into two more sublattices a and b consisting of even and odd sites, similarly for sublattice $\beta$ which splits into c and d.

From the previous sections, we already know what becomes of the terms $c^\dagger_{k_x, k_y, \alpha}c_{k_x, k_y, \alpha}$ and $c^\dagger_{k_x, k_y, \beta}c_{k_x, k_y, \beta}$. To treat the mixed terms, we calculate
\begin{align*}
\sum_{- \pi \leq k < \pi} f(k) c^\dagger_{k, \alpha}c_{k, \beta} =&  \sum_{- \pi / 2 \leq k < \pi/2} \frac{1}{2} f(k) \left[ c^\dagger_{2k, a} + e^{ik} c^\dagger_{2k, b} \right]  \left[ c_{2k, c} + e^{-ik} c_{2k, d} \right] \\
&+ \sum_{\pi / 2 \leq k < \pi} \frac{1}{2} f(k) \left[ c^\dagger_{2k - 2 \pi, a} + e^{ik} c^\dagger_{2k - 2 \pi, b} \right]  \left[ c_{2k - 2 \pi, c} + e^{-ik} c_{2k - 2 \pi, d} \right] \\
&+ \sum_{- \pi \leq k < \pi/2} \frac{1}{2} f(k) \left[ c^\dagger_{2k + 2 \pi, a} + e^{ik} c^\dagger_{2k + 2 \pi, b} \right]  \left[ c_{2k + 2 \pi, c} + e^{-ik} c_{2k + 2 \pi, d} \right] \\
 =&  \sum_{-\pi \leq \tilde k < \pi} \frac{1}{2}  f(\tilde k / 2) \left[ c^\dagger_{\tilde k, a} + e^{i \tilde k / 2} c^\dagger_{\tilde k, b} \right]  \left[ c_{\tilde k, c} + e^{-i \tilde k / 2} c_{\tilde k, d} \right] \\
&+ \sum_{- \pi \leq \tilde k < 0} \frac{1}{2} f(\tilde k / 2 + \pi) \left[ c^\dagger_{\tilde k, a} + e^{i (\tilde k / 2 + \pi)} c^\dagger_{\tilde k \pi, b} \right]  \left[ c_{\tilde k \pi, c} + e^{-i(\tilde k / 2 + \pi)} c_{\tilde k, d} \right] \\
&+ \sum_{0 \leq \tilde k < \pi} \frac{1}{2} f(\tilde k / 2 - \pi) \left[ c^\dagger_{\tilde k, a} + e^{i(\tilde k / 2 - \pi)} c^\dagger_{\tilde k, b} \right]  \left[ c_{\tilde k, c} + e^{-i(\tilde k / 2 - \pi)} c_{\tilde k, d} \right].
\end{align*}
Again, we use that $f(k)$ is $2 \pi$-periodic and write
\begin{align*}
\sum_{- \pi \leq k < \pi} f(k) c^\dagger_{k, \alpha}c_{k, \beta} =& \frac{1}{2}   \sum_{-\pi \leq \tilde k < \pi} \left \{ \left[f(\tilde k / 2) + f(\tilde k / 2 + \pi) \right]  \left[ c^\dagger_{\tilde k, a}c_{\tilde k, c} + c^\dagger_{\tilde k, b}c_{\tilde k, d} \right] \right. \\
& \left . + \left[f(\tilde k / 2) - f(\tilde k / 2 + \pi) \right] \left[e^{-i\tilde k / 2} c^\dagger_{\tilde k, a}c_{\tilde k, d} + e^{i\tilde k / 2} c^\dagger_{\tilde k, b}c_{\tilde k, c} \right] \right \} 
\end{align*}
Now we can write down the backfolded Hamiltonian

\begin{align*}
H_\text{2d, two band} =& \sum_{k_x, k_y} \left [ \underbrace{\left[d_0(k_x, k_y) +  d_z(k_x, k_y) \right]}_{f_1(k_x, k_y)} c^\dagger_{k_x, k_y, \alpha} c_{k_x, k_y, \alpha} \right . \\
& + \left( \underbrace{\left[ d_x(k_x, k_y) - i d_y(k_x, k_y) \right ]}_{f_2(k_x, k_y)} c^\dagger_{k_x, k_y, \alpha} c_{k_x, k_y, \beta} + \text{H.c.} \right )\\
&+ \left . \underbrace{\left [d_0(k_x, k_y) - d_z(k_x, k_y)\right]}_{f_3(k_x, k_y)} c^\dagger_{k_x, k_y, \beta} c_{k_x, k_y, \beta} \right] \\
=& \sum_{k_x, \tilde k_y} \left(c^\dagger_{k_x, \tilde k_y, a}, c^\dagger_{k_x, \tilde k_y, b}, c^\dagger_{k_x, \tilde k_y, c}, c^\dagger_{k_x, \tilde k_y, d}\right) 
\begin{bmatrix}
B_1(k_x, \tilde k_y) & B_2(k_x, \tilde k_y)\\
B_2^\dagger(k_x, \tilde k_y) & B_3(k_x, \tilde k_y)
\end{bmatrix}
\begin{pmatrix}
c_{k_x, \tilde k_y, a} \\
c_{k_x, \tilde k_y, b} \\
c_{k_x, \tilde k_y, c} \\
c_{k_x, \tilde k_y, d}
\end{pmatrix}
\end{align*}
where the blocks read:
\begin{align*}
B_1(k_x, \tilde k_y) =& \frac{1}{2}
\begin{bmatrix}
 [f_1(k_x, \tilde k_y / 2) + f_1(k_x, \tilde k_y / 2 + \pi)]  &  e^{-i\tilde k_y / 2}[f_1(k_x, \tilde k_y / 2) - f_1(k_x, \tilde k_y / 2 + \pi)] \\
 e^{i\tilde k_y / 2}[f_1(k_x, \tilde k_y / 2) - f_1(k_x, \tilde k_y / 2 + \pi)] &  [f_1(k_x, \tilde k_y / 2) + f_1(k_x, \tilde k_y / 2 + \pi)] 
\end{bmatrix} \\
B_2(k_x, \tilde k_y) =& \frac{1}{2}
\begin{bmatrix}
 [f_2(k_x, \tilde k_y / 2) + f_2(k_x, \tilde k_y / 2 + \pi)]  &  e^{-i\tilde k_y / 2}[f_2(k_x, \tilde k_y / 2) - f_2(k_x, \tilde k_y / 2 + \pi)] \\
 e^{i\tilde k_y / 2}[f_2(k_x, \tilde k_y / 2) - f_2(k_x, \tilde k_y / 2 + \pi)] &  [f_2(k_x, \tilde k_y / 2) + f_2(k_x, \tilde k_y / 2 + \pi)] 
\end{bmatrix} \\ 
B_3(k_x, \tilde k_y) =& \frac{1}{2}
\begin{bmatrix}
 [f_3(k_x, \tilde k_y / 2) + f_3(k_x, \tilde k_y / 2 + \pi)]  &  e^{-i\tilde k_y / 2}[f_3(k_x, \tilde k_y / 2) - f_3(k_x, \tilde k_y / 2 + \pi)] \\
 e^{i\tilde k_y / 2}[f_3(k_x, \tilde k_y / 2) - f_3(k_x, \tilde k_y / 2 + \pi)] &  [f_3(k_x, \tilde k_y / 2) + f_3(k_x, \tilde k_y / 2 + \pi)] 
\end{bmatrix}
\end{align*}
We may write the back-folded Bloch Hamiltonian as 
\begin{align*}
H(k_x, \tilde k_y) =& \frac{1}{2} [d_0(k_x, \tilde k_y / 2) + d_0(k_x, \tilde k_y / 2 + \pi) ] \sigma^0 \otimes \sigma^0 \\
& + \frac{1}{2}  \cos(\tilde k_y /2) [d_0(k_x, \tilde k_y / 2) - d_0(k_x, \tilde k_y / 2 + \pi) ] \sigma^0 \otimes \sigma^x \\
& + \frac{1}{2}  \sin(\tilde k_y /2) [d_0(k_x, \tilde k_y / 2) - d_0(k_x, \tilde k_y / 2 + \pi) ] \sigma^0 \otimes \sigma^y \\
&+  \frac{1}{2} [d_z(k_x, \tilde k_y / 2) + d_z(k_x, \tilde k_y / 2 + \pi) ] \sigma^z \otimes \sigma^0 \\
& + \frac{1}{2}  \cos(\tilde k_y /2) [d_z(k_x, \tilde k_y / 2) - d_z(k_x, \tilde k_y / 2 + \pi) ] \sigma^z \otimes \sigma^x \\
& + \frac{1}{2}  \sin(\tilde k_y /2) [d_z(k_x, \tilde k_y / 2) - d_z(k_x, \tilde k_y / 2 + \pi) ] \sigma^z \otimes \sigma^y \\
&+  \frac{1}{2} [d_x(k_x, \tilde k_y / 2) + d_x(k_x, \tilde k_y / 2 + \pi) ] \sigma^x \otimes \sigma^0 \\
&+ \frac{1}{2}  \cos(\tilde k_y /2) [d_x(k_x, \tilde k_y / 2) - d_x(k_x, \tilde k_y / 2 + \pi) ] \sigma^x \otimes \sigma^x \\
&+ \frac{1}{2}  \sin(\tilde k_y /2) [d_x(k_x, \tilde k_y / 2) - d_x(k_x, \tilde k_y / 2 + \pi) ] \sigma^x \otimes \sigma^y \\
&+  \frac{1}{2} [d_y(k_x, \tilde k_y / 2) + d_y(k_x, \tilde k_y / 2 + \pi) ] \sigma^y \otimes \sigma^0 \\
& + \frac{1}{2}  \cos(\tilde k_y /2) [d_y(k_x, \tilde k_y / 2) - d_y(k_x, \tilde k_y / 2 + \pi) ] \sigma^y \otimes \sigma^x\\
& + \frac{1}{2}  \sin(\tilde k_y /2) [d_y(k_x, \tilde k_y / 2) - d_y(k_x, \tilde k_y / 2 + \pi) ] \sigma^y \otimes \sigma^y
\end{align*}


\section{Toy model}
Let us set
\begin{align*}
d_0 =& 0, \\
d_x =& \sin(k_x), \\
d_y =& \sin(k_y), \\
d_z =& r - \cos(k_x) - \cos(2 k_y).
\end{align*}
The backfolded Hamiltonian is then

\begin{align*}
H(k_x, \tilde k_y) =& \sin(k_x) \sigma^x \otimes \sigma^0 + \sin(\tilde k_y/2) \sigma^y \otimes \left[\cos(\tilde k_y/2) \sigma^x +  \sin(\tilde k_y/2)\sigma^y \right ] \\
& +  \left[r - \cos(k_x) - \cos(\tilde k_y) \right] \otimes \sigma^z \otimes \sigma^0
\end{align*}

\subsection{Real space}
We use that 
\begin{align*}
c^\dagger_{\bm{k}, \gamma}  = \frac{1}{\sqrt{N}} \sum_{\bm j } e^{i \bm k \cdot \bm j}  c^\dagger_{\bm j, \gamma}
\end{align*}
and
\begin{align*}
\sum_{\bm j } c^\dagger_{\bm{j}, \gamma} c_{\bm{j} + \bm \delta, \gamma'}   =& \frac{1}{N} \sum_{\bm j} \sum_{\bm k_1} \sum_{\bm k_2}  e^{-i \bm k_1 \cdot \bm j }   e^{i \bm k_2 \cdot (\bm j + \bm \delta) }  c^\dagger_{\bm k_1, \gamma} c_{\bm k_2, \gamma'} \\
=& \sum_{\bm k} e^{i \bm k \bm \delta} c^\dagger_{\bm k, \gamma} c_{\bm k, \gamma'}
\end{align*}
to derive the real-space expression for a ribbon geometry with \gls{obc} along x.
\printnoidxglossaries

\newpage
\begin{thebibliography}{10}
\bibitem{Bosonization_OBC}
M. Fabrizio and A. O. Gogolin, {\em Interacting one-dimensional electron gas with open boundaries}, \href{https://journals.aps.org/prb/abstract/10.1103/PhysRevB.51.17827}{Phys. Rev. B {\bfseries 51}, 17827 (1995)}.


\bibitem{Bosonization_Delft}
J. v. Delft and H. Schoeller, {\em Bosonization for Beginners — Refermionization for Experts}, \href{https://arxiv.org/abs/cond-mat/9805275}{ arXiv:cond-mat/9805275 (1998)}.

\bibitem{Giamarchi}
T. Giamarchi, {\em Quantum Physics in One Dimension}, Clarendon Press, Oxford (2003).

\bibitem{Schoenhammer}
K. Schoenhammer, {\em Interaction fermions in one dimension: The Tomonaga-Luttinger model}, \href{https://arxiv.org/abs/cond-mat/9710330}{ arXiv:cond-mat/9710330v3 (1998)}.

\bibitem{Haldane}
F. D. M. Haldane, {\em 'Luttinger liquid theory' of one-dimensional quantum fluids. I. Properties of the Luttinger model and their extension to the general 1D interacting spinless Fermi gas},
\href{https://iopscience.iop.org/article/10.1088/0022-3719/14/19/010/meta}{J. Phys. C: Solid State Phys. {\bfseries 14} 2585 (1981)}.

\bibitem{Haldane_2}
F. D. M. Haldane, {\em Luttinger's Theorem and Bosonization of the Fermi Surface}, \href{https://arxiv.org/abs/cond-mat/0505529}{arXiv:cond-mat/0505529v1 (2005)}.

\bibitem{Halperin_Sarma}
Jay D. Sau, B. I. Halperin, K. Flensberg, and S. D. Sarma, {\em Number conserving theory for topologically protected degeneracy in one-dimensional fermions}, \href{https://journals.aps.org/prb/abstract/10.1103/PhysRevB.84.144509}{Phys. Rev. B {\bfseries 84}, 144509 (2011)}.

\bibitem{Zoller_model}
C. V. Kraus, M. Dalmonte, M. A. Baranov, A. M. Läuchli, and P. Zoller, {\em Majorana Edge States in Atomic Wires Coupled by Pair Hopping}, \href{https://journals.aps.org/prl/abstract/10.1103/PhysRevLett.111.173004}{Phys. Rev. Lett. {\bfseries 111}, 173004 (2013)}.

\bibitem{Fidkowski_et_al}
L. Fidkowski, R. M. Lutchyn, C. Nayak, and M. P. A. Fisher, {\em Majorana zero modes in one-dimensional quantum wires without
long-ranged superconducting order}, \href{https://journals.aps.org/prb/abstract/10.1103/PhysRevB.84.195436}{Phys. Rev. B {\bfseries 84}, 195436 (2011)}.

\bibitem{Hong_Hao}
M. Cheng and H.-H. Tu, {\em Majorana edge states in interacting two-chain ladders of fermions}, \href{https://journals.aps.org/prb/abstract/10.1103/PhysRevB.84.094503}{Phys. Rev. B {\bfseries 84}, 094503 (2011)}.

\bibitem{trigonometric_identites}
A. Arnal, F. Casas, C. Chiralt, {\em A note on trigonometric identities involving non-commuting matrices}, \href{https://arxiv.org/abs/1702.06069}{arXiv:1702.06069v1 (2017)}.

\end{thebibliography}



\end{document}

