\documentclass[aps,prb,amsmath,amssymb,twocolumn, superscriptaddress]{revtex4-2}

\usepackage{graphicx}
\usepackage{adjustbox}
\usepackage{bm}
\usepackage{color}
\usepackage{braket}
\usepackage{standalone}
\usepackage{multirow}
\usepackage{tikz}
\usepackage{mathrsfs}
\usepackage[utf8]{inputenc}
\usepackage{dsfont}
\usepackage[colorlinks,bookmarks=true,citecolor=blue,linkcolor=red,urlcolor=blue]{hyperref}

\newcommand{\JCB}[1]{{\color{green} #1}}
\newcommand{\BM}[1]{{\color{orange} #1}}
\newcommand{\eq}[1]{Eq.~(\ref{#1})}
\newcommand{\affTUD}{Institute of Theoretical Physics${\rm ,}$ Technische Universit\"{a}t Dresden and W\"{u}rzburg-Dresden Cluster of Excellence ct.qmat${\rm ,}$ 01062 Dresden${\rm ,}$ Germany}
\newcommand{\affMPIKS}{Max Planck Institute for the Physics of Complex Systems, N\"{o}thnitzer Str. 38, 01187 Dresden, Germany}



\usepackage{bbm}
\usepackage[subrefformat = parens, caption = false, labelformat=parens]{subfig}

\usepackage{floatrow}
\usepackage{verbatim}

\captionsetup{labelfont=normalfont,
	justification=raggedright,
	format=plain}
	
\usepackage[acronym]{glossaries}
\makenoidxglossaries

\newacronym{iqh}{IQH}{integer quantum Hall}
\newacronym{pbc}{PBC}{periodic boundary conditions}
\newacronym{obc}{OBC}{open boundary conditions}
\newacronym{dos}{DOS}{density of states}
\newacronym{gf}{GF}{Green's function}
\newacronym{kpm}{KPM}{kernel polynomial method}
\newacronym{wrt}{w.r.t.}{with respect to}

\glsdisablehyper

\begin{document}

\title{Supplemental Online Material: Quantum Hall Effect without Chern Bands}
\author{Benjamin Michen}
\email{benjamin.michen@tu-dresden.de}
\affiliation{\affTUD}
\author{Jan Carl Budich}
\email{jan.budich@tu-dresden.de}
\affiliation{\affTUD}
\affiliation{\affMPIKS}

\date{\today}


\maketitle

%\appendix

\onecolumngrid

\section{Details on the model}
The translation-invariant part of the model from the main text is given by 
\begin{align}
\hat H_0 = \sum_{\bm k} \bm c_{\bm k}^\dagger h_0(\bm k) \bm c_{\bm k}
\end{align}
with fermionic creation operators $\bm c_{\bm k}^\dagger = (c_{\bm k, a}^\dagger, c_{\bm k, b}^\dagger)$ in reciprocal space and the Bloch Hamiltonian 
\begin{align}
h_0(\bm k) =
\begin{pmatrix}
d_z(\bm k) & d_x(\bm k) - i d_y(\bm k) \\
d_x(\bm k) + i d_y(\bm k) & - d_z(\bm k)
\end{pmatrix},
\end{align}
where 
\begin{align}
d_x(\bm k) =& \gamma  \sin(k_x), \nonumber \\
d_y(\bm k) =& \lambda(k_x)  \sin(k_y), \nonumber \\
d_z(\bm k) =&  \gamma_2[r - \cos(2 k_x)] - \lambda(k_x) \cos(k_y),
\end{align}
and $\lambda(k_x) = \epsilon_1 + \epsilon_2 \left[1 - \cos(k_x)\right]/2$. Using $\cos(x)\cos(y) = [\cos(x+y) + \cos(x-y)]/2$ and $\cos(x)\sin(y) = [\sin(x+y) - \sin(x-y)]/2$, we may rewrite this as
\begin{align}
d_x(\bm k) =& \gamma \sin(k_x), \nonumber \\
d_y(\bm k) =& (\epsilon_1 + \epsilon_2 / 2) \sin(k_y) -  \epsilon_2 [\sin(k_x + k_y) - \sin(k_x - k_y)] / 4, \nonumber \\
d_z(\bm k) =& \gamma_2[r - \cos(2 k_x)] - (\epsilon_1 + \epsilon_2 / 2)\cos(k_y) + \epsilon_2 [\cos(k_x + k_y) + \cos(k_x - k_y)] / 4. 
\end{align}

\subsection{Real space Hamiltonian}
The formulation in real space following from the Fourier transform $c_{k, \alpha}^\dagger = \frac{1}{\sqrt{N_x N_y}} \sum_{\bm j} e^{i \bm k \cdot \bm j} c_{\bm j , \alpha}^\dagger$ and the relation 
\begin{align}
\sum_{\bm j } c^\dagger_{\bm{j}, \gamma} c_{\bm{j} + \bm \delta, \gamma'} =& \sum_{\bm k} e^{i \bm k \bm \delta} c^\dagger_{\bm k, \gamma} c_{\bm k, \gamma'},
\end{align}
as
\begin{align}
\hat H_{0} = & \sum_{\bm j} \left [\gamma_2 r \bm c_{\bm{j}}^\dagger \sigma^z \bm c_{\bm{j }} + \left \{\frac{\gamma}{2i} \bm c_{\bm{j}}^\dagger \sigma^x \bm c_{\bm{j + \delta_x}} + \frac{\epsilon_1 + \epsilon_2 / 2}{2i} \bm c_{\bm{j}}^\dagger \sigma^y \bm c_{\bm{j + \delta_y}}  - \frac{\epsilon_2}{8i} \bm c_{\bm{j}}^\dagger \sigma^y \bm c_{\bm{j + \delta_x + \delta_y}}  + \frac{\epsilon_2}{8i} \bm c_{\bm{j}}^\dagger \sigma^y \bm c_{\bm{j + \delta_x - \delta_y}} \right. \right. \nonumber \\
& - \left. \left. \frac{\epsilon_1 + \epsilon_2 / 2}{2} \bm c_{\bm{j}}^\dagger \sigma^z \bm c_{\bm{j  + \delta_y}} + \frac{\epsilon_2}{8} \bm c_{\bm{j}}^\dagger \sigma^z \bm c_{\bm{j  + \delta_x + \delta_y}} +  \frac{\epsilon_2}{8} \bm c_{\bm{j}}^\dagger \sigma^z \bm c_{\bm{j  + \delta_x - \delta_y}} - \frac{\gamma_2}{2} \bm c_{\bm{j}}^\dagger \sigma^z \bm c_{\bm{j  + 2 \delta_x}} + \text{H.c.}\right\} \right ].
\end{align}

\subsection{Berry curvature and energies}
The energies of the upper and lower band, which we denote here by $\pm$, are given by 
\begin{align}
E_\pm = \pm |\bm d| =& \sqrt{[\gamma \sin(k_x)]^2 + [\lambda(k_x) \sin(k_y)]^2 + [\gamma_2[r - \cos(2 k_x)] - \lambda(k_x) \cos(k_y)]^2}. \label{Eqn:Energy_analytically_App}
\end{align}
The Berry curvature for each band can be expressed as
\begin{align}
\mathcal F_{\pm}(\bm k) = \mp \bm {\hat  d} \cdot [\partial_{k_x} \bm {\hat  d} \times \partial_{k_y} \bm {\hat  d}]/ 2 = \frac{\mp 1}{|\bm d|^3}\bm {d} \cdot [\partial_{k_x} \bm {d} \times \partial_{k_y} \bm {d}]/ 2, \label{Eqn:Berry_curvature_analytically_App}
\end{align}
where $\bm {\hat  d} = \bm d / |\bm d|$ denotes the unit vector along $\bm d$. With
\begin{align}
\partial_{k_x} \bm d =& 
\begin{pmatrix}
\gamma \cos(k_x) \\
[\epsilon_2 / 2]\sin(k_x) \sin(k_y) \\
2 \gamma_2 \sin(2k_x) - [\epsilon_2 / 2]\sin(k_x) \cos(k_y) 
\end{pmatrix}, \quad 
\partial_{k_y} \bm d = 
\begin{pmatrix}
0 \\
\lambda(k_x)\cos(k_y) \\
\lambda(k_x) \sin(k_y)
\end{pmatrix},
\end{align}
we find 

\begin{align}
\partial_{k_x} \bm {\hat  d} \times \partial_{k_y} \bm {\hat  d} = 
\begin{pmatrix}
[\epsilon_2 / 2] \lambda(k_x) \sin(k_x)[\sin^2(k_y) + \cos^2(k_y)] - 2 \gamma_2 \lambda(k_x) \sin(2k_x) \cos(k_y) \\
- \gamma \lambda(k_x) \cos(k_x) \sin(k_y) \\
\gamma \lambda(k_x) \cos(k_x) \cos(k_y)
\end{pmatrix}
\end{align}
and ultimately
\begin{align}
\bm {d} \cdot [\partial_{k_x} \bm {d} \times \partial_{k_y} \bm {d}] =& \lambda(k_x) \{ [\epsilon_2 \gamma / 2] \sin^2(k_x) - 2 \gamma \gamma_2 \sin(k_x) \sin(2k_x) \cos(k_y)\nonumber  \\
&- \gamma \lambda(k_x) \cos(k_x) \sin^2(k_y) \nonumber\\
&+ \gamma \gamma_2 [r - \cos(2 k_x)] \cos(k_x) \cos(k_y) -  \gamma \lambda(k_x) \cos(k_x) \cos^2(k_y)\} \nonumber\\
=& \lambda(k_x) \{[\epsilon_2 \gamma / 2] \sin^2(k_x) - 2 \gamma \gamma_2 \sin(k_x) \sin(2k_x) \cos(k_y)  \nonumber \\
&+ \gamma \gamma_2 [r - \cos(2 k_x)] \cos(k_x) \cos(k_y) - \gamma \lambda(k_x) \cos(k_x) \}.
\end{align}
With this, we have an analytic expression for \eq{Eqn:Berry_curvature_analytically_App}, which we can be evaluated on a discrete lattice in the First Brillouin zone together with the energy \eq{Eqn:Energy_analytically_App}. To obtain the $\sigma_{xy}^{W=0}(E_\mathrm{F})$ curve in the first figure of the main text in a computationally efficient manner, the resulting array can simply be sorted by energy and then summed up.

\subsection{Non-topological edge states in Cylinder geometry}
We find that the separation of topological charges within the bands can cause a pair of edges states to emerge at a boundary. However, this also depends on the direction of the boundary. Importantly, if an edge state emerges, it will always have an anti-chiral partner due to the vanishing total Chern number of the bands, i.e. it is not topologically protected and can be gapped out continuously. 

To illustrate this, we compute the spectrum with \gls{pbc} in $x$-direction and \gls{obc} in $y$-direction in Fig.~\subref*{Fig:Cylinder_spec_a}. The edge states are highlighted by colors. The condition to be considered an edge state at the left / right edge is that $90\%$ of the wave function amplitude is concentrated within the left / right $25\%$ of the system. By contrast, in the spectrum with 
\gls{pbc} in $y$-direction and \gls{obc} in $x$-direction presented in 
Fig.~\subref*{Fig:Cylinder_spec_b}, no such states are present.

\begin{figure}[htp]	 
{
    \vbox to 0pt {
            \raggedright
            \textcolor{white}{
                \subfloatlabel[1][Fig:Cylinder_spec_a]
                \subfloatlabel[2][Fig:Cylinder_spec_b]
            }
        }
}
{\includegraphics[trim={0.75cm 0cm 0.75cm 0.cm}, width=0.80\linewidth]{Fig_1_App_publication.png}}
\caption{Spectra for cylinders with a height of 100 unit cells for parameters $r = 1.5$, $\epsilon_1 = 0.3$, $\epsilon_2 = 2$, $\gamma  =2$, and $\gamma_2 = 0.3$ as a function of momentum in the circumferential direction. Edge states indicated if present. a) Result for \gls{obc} in $y$-direction, a pair of edge states emerges. b) Result for \gls{obc} in $x$-direction, no edge states present.
 }\label{Fig:Edge_states}
\end{figure}

\section{Additional data on the four terminal Hall conductance}
To obtain the curve for the Hall conductance $\sigma_{xy}$ presented in Fig.~1 of the main text, we model the system as a four terminal scattering experiment with KWANT \cite{Kwant_App, Kwant_Mumps_App}. Following Refs.~ \cite{Datta_App, 4_terminal_Hall_App}, we extract the elements of the conductance tensor from the scattering amplitudes as 
\begin{align}
\sigma_{xx}  = \frac{e^2}{h} T_\mathrm{R \leftarrow L}, \quad \sigma_{xy}  = \frac{e^2}{h} (T_\mathrm{L \leftarrow T} - T_\mathrm{L \leftarrow B}), \quad \sigma_{yx}  = \frac{e^2}{h} (T_\mathrm{B \leftarrow R} - T_\mathrm{B \leftarrow L}), \quad \sigma_{yy}  = \frac{e^2}{h} T_\mathrm{T \leftarrow B}.
\end{align}
Here, $T_{p \leftarrow q}$ with $p, q = $ L (left), R (right), B (bottom), T (top) denotes the transmission amplitude from terminal $q$ to $p$. In Fig.~\ref{Fig:Hall_cond_4_terminal}, we present the result for all elements of the conductance tensor. The quantized plateaus of the off-diagonal elements are accompanied by a vanishing of $\sigma_{xx}$ and $\sigma_{yy}$. The peaks of the diagonal entries in the transition region between the Hall plateaus are indicative of a quantum phase transition. We illustrate the standard deviation of as a corridor (in Fig.~1 of the main text, we use the standard error which is smaller by a factor of $1 / \sqrt{S}$ for visual clarity).

We also present a complementary way of extracting the Hall conductance from the four terminal scattering data. If all the transmission amplitudes $T_{p \leftarrow q}$ are known, the conductance matrix can be obtained as 
\begin{align}
G_{p, q} =  - \frac{e^2}{h} T_{p \leftarrow q} + \delta_{p, q} \frac{e^2}{h} \sum_{q'} T_{q' \leftarrow p}.
\end{align}
The units and sign conventions are chosen such that the current at terminal $p$ is obtained as $I_p = \sum_{q} G_{p,q} V_q$, where $V_q = -e \mu_q$ is the Voltage at terminal $q$, obtained from the chemical potential $\mu_q$ and the elementary charge $e = 1.602176634 \text{\textsc{e-}}19\mathrm{C}$.  

As a complementary approach, given the conductance matrix it is possible to impose a current in one direction and determine the transverse voltage by solving the linear equation $\bm I = G \bm V$. For $\sigma_{xy}$, this amounts to setting $I_L = I_x$, $I_R = -I_x$, $I_B = 0$, $I_T = 0$ with some arbitrary value $I_x$. This leaves the equation $\bm I = G \bm V$ underdetermined, since the current does not change under a constant shift of all voltages. We can thus set, e.g., $V_T = 0$, which amounts to truncating the conductance matrix to the top left $3\times3$ block and solving for $V_L$, $V_R$, and $V_B$. The Hall voltage is then $V_\mathrm{Hall} = V_B - V_T = V_B$ and the Hall conductance is found to be $\sigma_{xy} = I_x / V_\mathrm{Hall} = I_x / V_B$. To determine $\sigma_{yx}$, the same the same procedure is used for the currents $I_L = 0$, $I_R = 0$, $I_B = I_y$, $I_T = - I_y$. This approach only yields a definite result for $E_\mathrm{F} $ inside the conductance plateaus, where the bulk of the system is insulating. There, however, it resolves the Hall plateaus very sharply and with small variance as we show in Fig.~\ref{Fig:V_Hall_4_terminal}.


\begin{figure}[htp]	 
\includegraphics[trim={0.75cm 0cm 0.75cm 0.cm}, width=0.95\linewidth]{Fig_2_App_publication.png}
\caption{Elements of the conductance tensor for a system size of $N_x = 2000$, $N_y = 500$ and parameters $r = 1.5$, $\epsilon_1 = 0.3$, $\epsilon_2 = 2$, $\gamma  =2$, $\gamma_2 = 0.3$ with disorder strength $W = 1.5$ as obtained from a standard four terminal scattering calculation. The leads attached to the vertical faces have a width of $50$ unit cells and those at the horizontal faces have a width of $200$ unit cells. The result is the average of $S = 40$ disorder realizations, standard deviation is indicated as a corridor.
 }\label{Fig:Hall_cond_4_terminal}
\end{figure}


\begin{figure}[htp]	 
\includegraphics[trim={0.75cm 0cm 0.75cm 0.cm}, width=0.95\linewidth]{Fig_3_App_publication.png}
\caption{Result for $\sigma_{xy}$ and $\sigma_{yx}$ for a system size of $N_x = 2000$, $N_y = 500$ and parameters $r = 1.5$, $\epsilon_1 = 0.3$, $\epsilon_2 = 2$, $\gamma  =2$, $\gamma_2 = 0.3$ with disorder strength $W = 1.5$. The leads attached to the vertical faces have a width of $50$ unit cells and those at the horizontal faces have a width of $200$ unit cells.  We present the result obtained from extracting the Hall voltage from the conductance matrix in blue and compare it to the result obtained from the difference of the scattering amplitudes in red (same data as Fig.~\ref{Fig:Hall_cond_4_terminal}). The standard deviation of the blue curve is indicated as a corridor.
 }\label{Fig:V_Hall_4_terminal}
\end{figure}


\section{Kernel Polynomial Approach to the Kubo Bastin formula} 
In the interest of a self-contained presentation, we briefly review the \gls{kpm} approach to computing the Kubo-Bastin formula that is implemented in the KWANT package, for more details see, e.g., Refs.~\cite{Kubo_Bastin_KPM_1_App, Kubo_Bastin_KPM_2_App, KPM_Review_App}.

\subsection{Chebyshev polynomial expansion}
The Chebyshev polynomials of the first kind are defined as
\begin{align}
T_m(x) = \cos [m \arccos(x)]
\end{align}
on the interval $[-1, 1]$ and obey the orthogonality relation
\begin{align}
\langle T_n , T_m \rangle_w = \frac{1 + \delta_{n,0}}{2} \delta_{n,m}
\end{align}
with respect to the scalar product defined by the weighting function $w(x) = [\pi \sqrt{1-x^2}]^{-1}$
\begin{align}
\langle f , g \rangle_w = \int_{-1}^1 f(x) g(x) w(x) dx.
\end{align}
Another useful property is the recursion relation
\begin{align}
T_{n+1}(x) = 2x T_n(x) - T_{n-1}(x). \label{Eqn:T_n_recursion}
\end{align}
For sufficiently regular functions on $[-1,1]$, an expansion in $T_n$ should converge uniformly and read
\begin{align}
f(x) = 2 \sum_{n = 0}^\infty \frac{\alpha_n T_n(x)}{\delta_{n,0} + 1}
\end{align}
with coefficients
\begin{align}
\alpha_n = \langle f, T_n \rangle_w =  \int_{-1}^1  f(x) T_n(x) w(x) dx. \label{Eqn:KPM_coefficients}
\end{align}
To obtain an approximation to $f(x)$, this expansion can be truncated at some order $M$
\begin{align}
f_M(x) = 2 \sum_{n = 0}^{M-1}  \frac{\alpha_n T_n(x)}{\delta_{n,0} + 1},
\end{align}
however naively doing so will usually lead to so-called Gibbs oscillations. To obtain a better approximation, it is possible to fold the truncated function with a Kernel
\begin{align}
f_{\text{KPM},M}(x) = \int_{-1}^1 \pi \sqrt{1 - y^2} f_M(y) K_M(x, y) dy, 
\end{align} 
where the Kernel $K_M(x, y)$ depends on the order $M$ of the approximation and can be chosen such that $f_{\text{KPM},M}(x)$ converges to $f(x)$ faster than the raw truncation $f_M(x)$. In practice, this amounts to multiplying the expansion coefficients by factors $g_n(M)$ that depend on the order M of the approximation:
\begin{align}
f_{\text{KPM},M}(x) = 2  \sum_{n = 0}^{M-1} g_n(M) \frac{\alpha_n T_n(x)}{\delta_{n,0} + 1}. \label{Eqn:KPM_expansion}
\end{align} 

The optimal choice of Kernel depends on the nature of the problem at hand and which properties of the approximated functions should be preserved to obtain an accurate solution. The calculations in this paper employ the Jackson Kernel, which is well-suited for most physics applications and amounts to the choice
\begin{align}
g_n = \frac{(M-n+1) \cos[\pi n / (M+1)] + \sin[\pi n / (M+1)] \cot[\pi / (M+1)]}{M+1}.
\end{align} 

\subsubsection{Extension to operator-valued functions}
To apply the method to operator-valued functions of the Hamiltonian $H$, it has to be rescaled such that its spectrum is contained within $[-1,1]$. This can be done by

\begin{align}
H \to \tilde H = \frac{2}{\Delta_E}\left(H - \frac{E^+ + E^-}{2} \right), \quad E \to \tilde E = \frac{2}{\Delta_E}\left(E - \frac{E^+ + E^-}{2} \right), \label{Eqn:rescaling_KPM}
\end{align}
where $\Delta_E = E^+ - E^-$ and $E^\pm$ denote upper / lower bounds on the spectrum that can be estimated by Krylow methods. Then, any function $f(\tilde E, \tilde H)$ of the rescaled Hamiltonian Hamiltonian and an energy $\tilde E$ can be expanded as
\begin{align}
f(\tilde  E,\tilde  H) =& \sum_l f(\tilde  E, \tilde  E_l) \ket{\tilde  E_l}\bra{\tilde  E_l} \nonumber \\
=& \sum_l \left[2 \sum_{n = 0}^{M-1}g_n(M) \alpha_n(\tilde E) T_n(\tilde E_l) \right ]\ket{\tilde E_l}\bra{\tilde E_l} \nonumber  \\
=& 2 \sum_{n = 0}^{M-1}\frac{g_n(M) \alpha_n(\tilde E) }{\delta_{n,0} + 1} T_n(\tilde H), \label{Eqn:KPM_expansion_operator}
\end{align} 
where we applied the KPM expansion \eq{Eqn:KPM_expansion} to the second argument of $f(\tilde E,\tilde E_l)$. Thus, the coefficients read (cf. \eq{Eqn:KPM_coefficients})

\begin{align}
\alpha_n(\tilde E) = \int_{-1}^{1} \frac{f(\tilde E, x) T_n(x)}{\pi \sqrt{1 -x^2}} dx. \label{Eqn:KPM_coefficients_operator}
\end{align} 

\subsection{Expansion of the Kubo-Bastin formula}
The exact linear response expression for the conductance tensor in a non-interacting system can be derived from the Kubo formula as 
\begin{align}
\sigma_{\alpha \beta}(E_\mathrm{F}, T) =& \frac{i e^2 \hbar}{V} \int_{E^-}^{E^+} \mathrm{d} E f_\mathrm{F}(E - E_\mathrm{F}, T) \mathrm{Tr} \left[v_\alpha \frac{\mathrm{d} G^+(E)}{\mathrm{d} E} v_\beta \rho(E)  - v_\alpha \rho(E) v_\beta  \frac{\mathrm{d} G^-(E)}{\mathrm{d} E} \right],\label{Eqn_App:Kubo_Bastin}
\end{align}
where $f_\mathrm{F}(E - E_\mathrm{F}, T) = 1 / (\exp[(E-E_\mathrm{F}) / (k_\mathrm{B} T)] + 1)$ \cite{Kubo_paper_App, Kubo_Bastin_App, Kubo_Bastin_modern_App}. Given the position operator $R_\alpha$, the velocity operators follow from the Heisenberg picture as $v_\alpha = \dot R_\alpha = -(i/ \hbar) [R_\alpha, H]$.

The other two relevant quantities are the \gls{dos} $\rho(E, H) = \delta(E - H)$ and the retarded/adavanced \gls{gf} $G^{\pm}(E, H) = [E \pm i 0^+ - H]^{-1}$. First of all, the above expression has to be rescaled in the sense of \eq{Eqn:rescaling_KPM}. To this end, note that $\delta((\tilde E + b) / a - H) = |a| \delta (\tilde E - (a H - b)) = a \delta (\tilde E - \tilde H)$ and $G^\pm[(\tilde E + b) / a, H] = [(\tilde E + b) / a - H + i 0^\pm]^{-1} = a  [\tilde E - \tilde H + i a 0^\pm]^{-1} = a G^\pm[\tilde E , \tilde H]$, where $a = 2/\Delta_E$, $b = (E^+ + E^-)/ \Delta_E$, from which we can infer by straightforward substitution
\begin{align}
\sigma_{\alpha \beta}(E_\mathrm{F}, T) =& \frac{i a^2 e^2 \hbar}{V} \int_{-1}^{1} \mathrm{d} \tilde E f_\mathrm{F}((\tilde E + b)/a - E_\mathrm{F}, T) \mathrm{Tr}\left[v_\alpha \frac{\mathrm{d} G^+(\tilde E)}{\mathrm{d} \tilde  E} v_\beta \rho(\tilde E) - v_\alpha \rho(\tilde  E) v_\beta \frac{\mathrm{d} G^- (\tilde E)}{\mathrm{d} \tilde E}   \right]. \label{Eqn_App:Kubo_Bastin_rescaled}
\end{align}
Now, the \gls{dos} and \gls{gf} can be expanded as per \eq{Eqn:KPM_expansion_operator}, with the coefficients following from \eq{Eqn:KPM_coefficients_operator}: 

\begin{align}
\alpha_{n, \rho}(\tilde E) =& \int_{-1}^{1} \frac{\delta(\tilde E-x) T_n(x)}{\pi \sqrt{1 -x^2}} dx = \frac{T_n(\tilde E)}{\pi \sqrt{1 - \tilde E^2}} \\
\alpha_{n, G^{\pm}}(\tilde E) =& \int_{-1}^{1} \frac{T_n(x)}{[\tilde E + i 0^\pm - x]\pi \sqrt{1 -x^2}} dx = \mathcal P \left[\int_{-1}^{1} \frac{T_n(x)}{\pi [\tilde E - x] \sqrt{1 -x^2}} dx\right]  \mp i \frac{T_n(\tilde E)}{\sqrt{1 - \tilde E^2}} \nonumber \\
=& U_{n-1}(\tilde E) \mp i \frac{T_n(\tilde E)}{\sqrt{1 - \tilde E^2}} = \mp i \frac{e^{\pm i n \arccos(\tilde E)}}{\sqrt{1 - \tilde E^2}}.
\end{align}
The real part of $\alpha_{n, G^{\pm}}(\tilde E)$ is a standard integral relation between the Chebyshev polynomials of the first kind, $T_n(x)$, and the second kind $U_n(x) = \sin[(n+1) \arccos(x)] / \sqrt{1 - x^2}$ \cite{KPM_Review_App}. The derivative of $\alpha_{n, G^{\pm}}(\tilde E)$ \gls{wrt} $\tilde E$ is
\begin{align}
\frac{d}{d \tilde E} \alpha_{n, G^{\pm}}(\tilde E) =& \mp i \left[ \frac{\tilde E}{(1 - \tilde E^2)^{3/2}} \mp \frac{i n}{1 - \tilde E^2}\right] e^{\pm i n \arccos(\tilde E)}.
\end{align}
Having obtained the coefficients, we can expand \eq{Eqn_App:Kubo_Bastin_rescaled} as
\begin{align}
\sigma_{\alpha \beta}(E_\mathrm{F}, T) =& \frac{4 a^2 e^2 \hbar}{\pi V} \int_{-1}^{1} \mathrm{d} \tilde  E \frac{f_\mathrm{F}((\tilde E + b)/a - E_\mathrm{F}, T) }{(1 - \tilde E^2)^2} \sum_{m,n = 0}^M \Gamma_{n,m}(\tilde E) \mu_{n,m}^{\alpha, \beta} , \label{Eqn_App:Kubo_Bastin_KPM}
\end{align}
where 
\begin{align}
\Gamma_{n,m}(\tilde E) = \left[\tilde E - i n \sqrt{1- \tilde E^2} \right] e^{i n \arccos(\tilde E)} T_{m} (\tilde E) + T_{n} (\tilde E) \left[\tilde E + i m \sqrt{1- \tilde E^2} \right] e^{- i m \arccos(\tilde E)} 
\end{align}
does not depend on the details of the system, which are contained in 
\begin{align}
\mu_{n, m}^{\alpha, \beta} = \frac{g_n(M) g_m(M)}{(1 + \delta_{n,0}) (1 + \delta_{m,0})} \mathrm{Tr}\left[v_\alpha T_{n}(H) v_\beta T_{m}(H)\right].
\end{align}
It is easily seen that $\Gamma_{m,n}^* = \Gamma_{n,m}$ and $(\mu_{m,n}^{\alpha, \beta})^* = \mu_{n,m}^{\alpha, \beta}$, which implies that \eq{Eqn_App:Kubo_Bastin_KPM} is a real number. The numerical cost is hidden in the evaluation of $\mu_{m,n}^{\alpha, \beta}$, however the trace of a generic operator $A$ can be sampled efficiently by using $R$ random phase vectors $\ket{\phi_l}$:
\begin{align}
\mathrm{Tr} [A] \approx \frac{1}{R} \sum_{l = 1}^R \bra{\phi_l} A \ket{\phi_l}.
\end{align}
This estimate improves with system size $N$, it can be shown that for a generic operator $A$ (in the sense of local and homogeneous support such that $\mathrm{Tr}[A^2] = O(N^2)$) the relative error should scale as $\sim \frac{1}{\sqrt{N R}}$  \cite{KPM_Review_App}. Furthermore, the recursion relation \eq{Eqn:T_n_recursion} can be used to evaluate the scalar products appearing in the approximation of the trace.

To obtain the data presented in the main text of the paper, we used the efficient implementation of the algorithm outlined here in the KWANT package \cite{Kwant_App, Kwant_Mumps_App}. To generate correct velocity operators for \gls{pbc}, we also acknowledge use of the extension KPM tools \cite{KPM_tools_App}.



\section{Hall conductance of the clean system from the Kubo-Bastin formula}
For completeness, we derive here that the Hall conductance of the clean system is indeed proportional to the accumulated Berry flux up to the Fermi energy $E_\mathrm{F}$. At $T = 0$, the Hall conductance according to the Kubo-Bastin Formula can be written as
\begin{align}
\sigma_{xy}(E_\mathrm{F}, T = 0) =& \frac{i e^2 \hbar}{V} \int_{- \infty}^{E_\mathrm{F}} \mathrm{d} E \mathrm{Tr} \left[v_x \frac{\mathrm{d} G^+(E)}{\mathrm{d} E} v_y \rho(E) - v_x \rho(E) v_y  \frac{\mathrm{d} G^-(E)}{\mathrm{d} E}  \right],
\end{align}
where all operators are to be interpreted as those arising from the single-particle tight-binding matrix \cite{Kubo_paper_App, Kubo_Bastin_App, Kubo_Bastin_modern_App}. In the clean, translation-invariant system, all appearing quantities are block-diagonal in the momentum basis. The blocks of the velocity operators are readily derived to be $v_\alpha(\mathrm k) = \partial_{k_\alpha} H(\bm k) / \hbar$,
the \gls{gf} terms become
\begin{align}
\frac{\mathrm{d} G^\pm(E, \bm k)}{\mathrm{d} E} = \frac{\mathrm{d} }{\mathrm{d} E} \sum_{l = 1}^{N_\mathrm{O}} \frac{1}{E + i 0^\pm - E_l(\bm k)} \ket{E_l(\bm k)} \bra{E_l(\bm k)} = - \sum_{l = 1}^{N_\mathrm{O}} \frac{1}{(E + i 0^\pm - E_l(\bm k))^2} \ket{E_l(\bm k)} \bra{E_l(\bm k)}, 
\end{align}
and the \gls{dos} 
\begin{align}
\rho(E, \bm k) = \sum_{l = 1}^{N_\mathrm{O}} \delta(E - E_l(\bm k)) \ket{E_l(\bm k)} \bra{E_l(\bm k)}.
\end{align}
The trace can be reduced to a sum over all $\bm k$ and a trace over the eigenstates $\ket{E_l(\bm k)}$ of the Bloch Hamiltonian at each $\bm k$ (even though there might be topological obstructions to finding a smooth gauge for the $\ket{E_l(\bm k)}$, this not a problem when taking the trace, as it is basis-independent). The Hall conductance thus becomes
 
\begin{align}
\sigma_{xy}(E_\mathrm{F}, T = 0) =& \frac{i e^2 }{V \hbar} \int_{- \infty}^{E_\mathrm{F}} \mathrm{d} E \sum_{\bm k} \sum_{l, l' = 1}^{N_\mathrm{O}} \left [ \bra{E_l(\bm k)}\left(\partial_{k_x} H(\bm k) \right)\ket{E_{l'}(\bm k)} \bra{E_{l'}(\bm k)} \left(\partial_{k_y} H(\bm k) \right)\ket{E_{l}(\bm k)}  \frac{- \delta(E - E_{l}(\bm k)) }{(E + i 0^+ - E_{l'}(\bm k))^2} \right .  \nonumber \\
& - \left.  \bra{E_l(\bm k)}\left(\partial_{k_x} H(\bm k) \right)\ket{E_{l'}(\bm k)} \bra{E_{l'}(\bm k)} \left(\partial_{k_y} H(\bm k) \right)\ket{E_{l}(\bm k)} \frac{- \delta(E - E_{l'}(\bm k)) }{(E + i 0^- - E_{l}(\bm k))^2} \right ].
\end{align}
The terms of the above equation where $l = l'$ vanish, for the rest we can neglect the $i 0^\pm$ regularization and carry out the integral to find

\begin{align}
\sigma_{xy}(E_\mathrm{F}, T = 0) =& -\frac{i e^2}{V \hbar} \sum_{\bm k} \sum_{l = 1}^{N_\mathrm{O}}\Theta(E_\mathrm{F} - E_l(\bm k))  \sum_{l' \neq l} \left [ \bra{E_l(\bm k)}\left(\partial_{k_x} H(\bm k) \right)\ket{E_{l'}(\bm k)} \bra{E_{l'}(\bm k)} \left(\partial_{k_y} H(\bm k) \right)\ket{E_{l}(\bm k)} \right .  \nonumber \\
& - \left.  \bra{E_l(\bm k)}\left(\partial_{k_y} H(\bm k) \right)\ket{E_{l'}(\bm k)} \bra{E_{l'}(\bm k)} \left(\partial_{k_x} H(\bm k) \right)\ket{E_{l}(\bm k)}  \right]  \frac{1}{(E_l(\bm k)  - E_{l'}(\bm k))^2} \nonumber \\
=& -\frac{ie^2}{V \hbar} \sum_{\bm k} \sum_{l = 1}^{N_\mathrm{O}}\Theta(E_\mathrm{F} - E_l(\bm k))  \sum_{l' \neq l} \left [ \braket{\partial_{k_x}  E_l(\bm k)|E_{l'}(\bm k)} \braket{E_{l'}(\bm k) |\partial_{k_y} E_{l}(\bm k)} \right . \nonumber \\
& - \left.  \braket{\partial_{k_y}  E_l(\bm k)|E_{l'}(\bm k)} \braket{E_{l'}(\bm k) |\partial_{k_x} E_{l}(\bm k)} \right] \nonumber \\
=& \frac{e^2}{V \hbar} \sum_{\bm k} \sum_{l = 1}^{N_\mathrm{O}}\Theta(E_\mathrm{F} - E_l(\bm k))   \left [-i (\braket{\partial_{k_x}  E_l(\bm k)|\partial_{k_y} E_{l}(\bm k)} - \braket{\partial_{k_y} E_l(\bm k)|\partial_{k_y} E_{l}(\bm k)}) \right ] \nonumber \\
=& \frac{e^2}{h} \sum_{l = 1}^{N_\mathrm{O}} \frac{1}{2 \pi} \int \mathrm{d}^2k  \Theta(E_\mathrm{F} - E_l(\bm k)) \mathcal F_l (\bm k) = \frac{e^2}{h} \sum_{l = 1}^{N_\mathrm{O}} \frac{1}{2 \pi} \int_{E_l(\bm k)<E_\mathrm{F}}\mathrm{d}^2k   \mathcal F_l (\bm k). \label{Eqn:Kubo_Bastin_clean}
\end{align}
For the second equality above, we used that $\bra{E_l(\bm k)}\left(\partial_{k_\alpha} H(\bm k) \right)\ket{E_{l'}(\bm k)} = \braket{\partial_{k_\alpha}  E_l(\bm k)|E_{l'}(\bm k)} [E_l(\bm k) - E_{l'}(\bm k)] = - \braket{E_l(\bm k)|\partial_{k_\alpha} E_{l'}(\bm k)} [E_l(\bm k) - E_{l'}(\bm k)]$ and for the third equation, we note that the term $l' = l$ yields zero and can formally be added to the sum $\sum_{l' \neq l}$ to yield a resolution of identity. Finally, the continuum limit is taken as $\Delta_{\bm k}^2 \sum{\bm k} = \int d^2 k$ with $\Delta_{\bm k}^2 = 4 \pi^2 / V$. 

\section{Additional data on the spectral localizer}
\BM{The spectral localizer serves as a type of local Chern marker and yields a topological index $Q$ [cf. Eq.~(7) of main text] as a function of spatial position and (Fermi) energy that remains valid in the presence of disorder. By construction, $Q$ is only well-defined in the presence of a localizer gap $g_L$ at the chosen energy [cf. Eq.~(6) of main text], which is only guaranteed to emerge in the presence of a spectral gap \cite{Spec_loc_5_App} or a mobility gap \cite{spec_loc_mobility_gap_App}. The physical significance of the topological index $Q$ is to predict the quantized value of the Hall conductance iff there is a spectral gap or a mobility gap that enforces a quantization. In the present work, our main focus is to evaluate the spectral localizer for energies in the bulk of the clean system (the red intervals in Fig.~\ref{Fig:Spec_Loc_Data_size}), where there is accordingly no guarantee for a finite localizer gap. Still, we observe a robust localizer gap $g_L$ for extended bulk energy windows and a non-zero value of $Q$ inside these windows. The purpose of this appendix is to provide numerical data to substantiate this claim and to make a rigorous argument for why the spectral localizer data of the clean system should correctly predict the quantized Hall conductance at the onset of disorder.}

We calculate the localizer gap for different system sizes and find that it saturates at $N_x  = N_y = N = 15$. The data is presented in Fig.~\ref{Fig:Spec_Loc_Data_size}. We note that the sign of the topological index $Q$ obtained from the spectral localizer is opposite to that of the observed Hall conductance in the presence of disorder, which is due to a different sign convention for the Berry curvature. The spectral localizer framework takes the Berry curvature as $F_n = i (\nabla \times \bra{n} \nabla \ket{n}) \cdot \bm e_z$ (see Eq.~(1) of Ref.~\cite{Spec_loc_5_App}) as opposed to the usual convention $F_n = -i (\nabla \times \bra{n} \nabla \ket{n} ) \cdot \bm e_z$ (see chapter three of Ref.~\cite{Bernevig_App}) used in the physics context. The latter should produce an equal sign of Chern number and Hall conductance as we also derive in \eq{Eqn:Kubo_Bastin_clean}. 

\begin{figure}[htp]	 
\includegraphics[trim={0.75cm 0cm 0.75cm 0.cm}, width=0.80\linewidth]{Fig_4_App_publication.png}
\caption{Localizer gap $g_L$ [cf. Eq.~(6) of main text] as a function of energy for parameters $r = 1.5$, $\epsilon_1 = 0.3$, $\epsilon_2 = 2$, $\gamma  =2$, and $\gamma_2 = 0.3$ for multiple 
system sizes of $N_x = N_y = N$. The scaling parameter is set to $\kappa = 0.4$. The value of $g_L$ is taken as the minimum of 100 reference positions inside the central Wigner-Seitz cell. Wherever the localizer gap is stable, we indicate the associated value of the topological index $Q$ [cf. Eq.~(7) of main text]. For reference, the clean bulk spectrum is shown in red and the energy window where $\sigma_{xy}^{W=0}(E_\mathrm{F})  > 0.5 e^2 / h$ [cf. Eq.~(2) of main text] is marked as a gray corridor. }\label{Fig:Spec_Loc_Data_size}
\end{figure}

We note that there is an extended energy window in Fig.~\ref{Fig:Spec_Loc_Data_size} where the localizer gap is zero. In principle, this is expected as the existence of a finite localizer gap is only guaranteed in the presence of a spectral gap \cite{Spec_loc_5_App} or a mobility gap \cite{spec_loc_mobility_gap_App}. For energies in the bulk of the clean system, there is neither a spectral nor a mobility gap and thus no guarantee for a conclusive statement on the topology of the system from the spectral localizer. However, our numerical data strongly indicates a stable localizer gap for certain energy windows in the bulk of the clean system. Assuming that this gap is indeed stable in the thermodynamic limit, the usual argument for the topological stability of the localizer gap and associated topological indices based on Weyl's inequality \cite{Spec_loc_5_App, Fine_structure_App} can be extended to the present situation. If we order the eigenvalues of two Hermitian operators $A$ and $B$ defined on an $n$-dimensional vector space as $\lambda_1  \geq \lambda_2 \geq ... \geq \lambda_n$, Weyl's inequality states that $\lambda_i(A) + \lambda_j(B) \leq  \lambda_{i + j - n}(A + B)$. By setting $j=1$ and $j=n$, it follows that $\lambda_1(A + B) - \lambda_1(B) \in [\lambda_n(B), \lambda_1(B)]$ and thus 

\begin{align}
|\lambda_i(A + B) - \lambda_i(A)| \leq ||B||_2,
\end{align}
where $||B||_2 = \text{max}\{|\lambda_1(B)|, |\lambda_n(B)|\}$ denotes the $L_2$ matrix norm, i.e. the maximum modulus of the eigenvalues. Setting $A = L^{\hat H_0}(x,y,E)$ and $B = L^{\hat H_0 + \hat W}(x,y,E) - L^{H_0}(x,y,E) = \sigma_z \otimes \hat W$, where we denote by $L^{\hat H}(x,y,E)$ the spectral localizer for a Hamiltonian $\hat H$, leads to 
\begin{align}
\left |\lambda_i\left[L^{\hat H_0 + \hat W}(x,y,E)\right] - \lambda_i \left[L^{\hat H_0}(x,y,E)\right] \right| \leq ||L^{\hat H_0 + \hat W}(x,y,E) - L^{H_0}(x,y,E)||_2 = ||\hat W ||_2.
\end{align}
This means that the change of any eigenvalue of $L^{\hat H_0 + \hat W}(x,y,E)$ relative to that of $L^{H_0}(x,y,E)$ is bounded by the maximum eigenvalue of $\hat W$. It follows for finite $g_{L^{H_0}}(E)$ that as long as $||\hat W ||_2 < g_{L^{H_0}}(E)$, the localizer gap in the bulk at energy $E$ cannot close in the presence of the disorder potential\BM{, and thus $Q$ remains unchanged}. However, a finite value of the disorder strength should still lead to Anderson localization and the formation of a mobility gap in the thermodynamic limit. \BM{Hence}, the topological localizer index $Q$ \BM{obtained at energies with finite $g_L$ in the clean system should agree with the expected quantization of the Hall conductance at the onset of disorder}, which is what we observe. 

To conclude this section, we would like to emphasize that we do not make any analytical argument for a finite localizer gap in the absence of a spectral or mobility gap, but only observe it as a numerical fact. It remains an interesting direction for future research whether mathematically rigorous statements can be made in this regard. 

\section{Relation to "Topological fine structure of an energy band"} 
In a recent work \cite{Fine_structure_App}, it has been observed that a trivial band that is coupled to two bands with Chern numbers $\pm 1$ can split in two non-trivial bands in the presence of disorder. This is also accompanied by a finite and non-trivial localizer gap inside the bulk energies of the clean system. However, a deeper physical reason for this remains elusive. We find that our present approach of relating the topological structure emerging from a trivial band in the presence of disorder to an energetic separation of Berry fluxes agrees well with the numerical observations made in Ref.~\cite{Fine_structure_App}.

Concretely, the model under investigation in Ref.~\cite{Fine_structure_App} is given by

\begin{align}
H(\bm k) = \begin{pmatrix} 
h_{11}(\bm k) & h_{12}(\bm k) & v \\
h_{12}(\bm k) ^*&  -h_{11}(\bm k) & 0 \\
v & 0 & 0 
\end{pmatrix} \label{Eqn:H_fine_structure}
\end{align}
with $h_{11}(\bm k) = 2(\cos(k_x) - \cos(k_y))$ and $h_{12}(\bm k) = \sqrt{2} e^{- i \pi /4} (e^{i k_x} + e^{i k_y} + i \left(e^{i (k_x + k_y)} + 1 \right)) $. The top left $2\times 2$ block describes two topological bands with Chern number $C= \pm 1$ and the term $v$ couples them to a third band centered at $E = 0$ that is trivial and completely flat for $v= 0$. A phase transition occurs at $v \approx 5.65$, where the $C = \pm 1$ bands simultaneously touch the middle band and revert to trivial bands. At small coupling $v$, transport calculations in the presence of disorder (also performed using KWANT) show that the middle band localizes while the top and bottom bands flow together and annihilate, which is the expected behavior. However, for $3.3 \lesssim v \lesssim 5.65$, the trivial middle band starts to split into two subbands that annihilate with the already present topological bands. Please see Ref.~\cite{Fine_structure_App} for data and details. 

To obtain the clean Hall conductance in the sense of \eq{Eqn:Kubo_Bastin_clean} of the Hamiltonian \eq{Eqn:H_fine_structure}, an efficient algorithm for calculating the Berry curvature such as Ref.~\cite{Chern_number_numerically_App} can be used. We present the result for different values of the coupling $v$ in Fig.~\ref{Fig:Fine_structure} and find that it predicts the behavior of the system under disorder observed in Ref.~\cite{Fine_structure_App} rather well. For $v = 2$, there is no energy in the central band where $\sigma_{xy} < 0.5 e^2 / h$, as Fig.~\subref*{Fig:Fine_structure_a} shows. Consequently, one would expect that disorder simply straightens out the line and leads to one big Hall plateau, which is equivalent to completely localizing the middle band. Moving to $v = 3.3$ in Fig.~\subref*{Fig:Fine_structure_b}, an energy window opens where $\sigma_{xy} < 0.5 e^2 / h$, in which a trivial phase with $\sigma_{xy} = 0$ should form at the onset of disorder. At $v = 4.5$ in Fig.~\subref*{Fig:Fine_structure_c}, the window $\sigma_{xy} < 0.5 e^2 / h$ broadens, suggesting a similar behavior to the case of $v = 3.3$. Finally, Fig.~\subref*{Fig:Fine_structure_d} shows the case of $v = 6.5$, which is past the quantum phase transition. There, $\sigma_{xy} < 0.5 e^2 / h$ everywhere, which is consistent with a complete localization of all bands in the presence of disorder. In summary, the expectations from the evaluation of the clean Hall conductance in Fig.~\ref{Fig:Fine_structure} fully agree with the numerical results of Ref.~\cite{Fine_structure_App}, which suggests that \eq{Eqn:Kubo_Bastin_clean} is a versatile tool to gain physical insight into the topological fine structure of energy bands.

\begin{figure}[htp]	 
{
    \vbox to 0pt {
            \raggedright
            \textcolor{white}{
                \subfloatlabel[1][Fig:Fine_structure_a]
                \subfloatlabel[2][Fig:Fine_structure_b]
                \subfloatlabel[3][Fig:Fine_structure_c]
                \subfloatlabel[4][Fig:Fine_structure_d]
            }
        }
}
{\includegraphics[trim={0.75cm 0cm 0.75cm 0.cm}, width=0.9\linewidth]{Fig_5_App_publication.png}}
\caption{Clean Hall conductance $\sigma_{xy}$ of the Hamiltonian \eq{Eqn:H_fine_structure} as a function of Fermi energy $E_\mathrm{F}$ for different coupling strengths $v$. The energy windows of the bulk bands are indicated as corridors for reference. The following description of subfigures states the value of $v$ and the the behavior observed in Ref.~\cite{Fine_structure_App} for the respective value. (a) $v = 2$, $\sigma_{xy}$ does not dip below  $0.5 e^2 / h$, in the central band. The central band does not split in the presence of disorder, but simply localizes. (b) $v = 3.3$, an energy window emerges where $\sigma_{xy} < 0.5 e^2 / h$. The central band begins to split and annihilate with the upper and lower band in the presence of disorder. (c) $v = 4.5$, the window where $\sigma_{xy} < 0.5 e^2 / h$ broadens and the response to disorder is similar to $v = 3.3$. (d) $v = 6.5$, the bands have touched and a phase transition has occured such that $\sigma_{xy} < 0.5 e^2 / h$ everywhere. All bands are trivial and simply localize at the onset of disorder.
 }\label{Fig:Fine_structure}
\end{figure}

\twocolumngrid

\begin{thebibliography}{10}

\bibitem[S1]{Datta_App}
S. Datta, \href{https://www.cambridge.org/core/books/electronic-transport-in-mesoscopic-systems/1E55DEF5978AA7B843FF70337C220D8B}{\em Electronic Transport in Mesoscopic Systems}, Cambridge (1995).

\bibitem[S2]{4_terminal_Hall_App}
G. Salerno, H. M. Price, M. Lebrat, S. Häusler, T. Esslinger, L. Corman, J.-P. Brantut, and N. Goldman, {\em Quantized Hall Conductance of a Single Atomic Wire: A Proposal Based on Synthetic Dimensions}, \href{https://journals.aps.org/prx/abstract/10.1103/PhysRevX.9.041001}{Phys. Rev. X {\bfseries 9}, 041001 (2019)}.

\bibitem[S3]{Kubo_Bastin_KPM_1_App}
J. H. Garcia, L.  Covaci, T. G. Rappoport, {\em Real-space calculation of the conductivity tensor for disordered topological matter}, \href{https://journals.aps.org/prl/abstract/10.1103/PhysRevLett.114.116602}{Phys. Rev. Lett. {\bfseries 114}, 116602 (2015)}.

\bibitem[S4]{Kubo_Bastin_KPM_2_App}
D. Ködderitzsch, K. Chadova, and H. Ebert, {\em Linear response Kubo-Bastin formalism with application to the anomalous and spin Hall effects: A first-principles approach}, \href{https://journals.aps.org/prb/abstract/10.1103/PhysRevB.92.184415}{Phys. Rev. B {\bfseries 92}, 184415 (2015)}.

\bibitem[S5]{KPM_Review_App}
A. Weisse, G. Wellein, A. Alvermann, H. Fehske, {\em The Kernel Polynomial Method}, \href{https://journals.aps.org/rmp/abstract/10.1103/RevModPhys.78.275}{Rev. Mod. Phys. {\bfseries 78}, 275 (2006)}.

\bibitem[S6]{Kubo_paper_App}
R. Kubo, {\em Statistical-Mechanical Theory of Irreversible Processes. I. General Theory and Simple Applications to Magnetic and Conduction Problems}, \href{https://journals.jps.jp/doi/10.1143/JPSJ.12.570}{Journal of the Physical Society of Japan {\bfseries 12}, pp. 570-586 (1957)}.

\bibitem[S7]{Kubo_Bastin_App}
A. Bastin, C. Lewinner, O. Betbeder-Matibet, and P.
Nozieres, {\em Quantum oscillations of the hall effect of a fermion gas with random impurity scattering}, \href{https://www.sciencedirect.com/science/article/pii/S0022369771801476}{ J. Phys. Chem. Solids {\bfseries 32}, 1811 (1971)}.

\bibitem[S8]{Kubo_Bastin_modern_App}
A. Cr\'epieux and P. Bruno, {\em Theory of the anomalous Hall effect from the Kubo formula and the Dirac equation}, \href{https://journals.aps.org/prb/abstract/10.1103/PhysRevB.64.014416}{Phys. Rev. B {\bfseries 64}, 014416 (2001)}.

\bibitem[S9]{Kwant_App} 
 C. W. Groth, M. Wimmer, A. R. Akhmerov, X. Waintal, {\em Kwant: a software package for quantum transport}, \href{https://iopscience.iop.org/article/10.1088/1367-2630/16/6/063065}{New J. Phys. 16, 063065 (2014)}.

\bibitem[S10]{Kwant_Mumps_App}
P. R. Amestoy, I. S. Duff, J. S. Koster, J. Y. L’Excellent, {\em Accelerating Optimization of Parametric Linear Systems by Model Order Reduction}, \href{https://epubs.siam.org/doi/10.1137/120869171}{SIAM. J. Matrix Anal. \& Appl. {\bf 23} (1), 15 (2014)}. 

\bibitem[S11]{KPM_tools_App}
To implement the Kubo-Bastin formula with PBC, we use the KWANT extension \href{https://kpm-tools.readthedocs.io/en/latest/index.html}{KPM Tools}, which is largely based on ideas from: 
D. Varjas, M. Fruchart, A. R. Akhmerov, P. M. Perez-Piskunow, {\em Computation of topological phase diagram of disordered $\mathrm{Pb}_{1-x} \mathrm{Sn}_x \mathrm{Te}$ using the kernel polynomial method}, \href{https://journals.aps.org/prresearch/abstract/10.1103/PhysRevResearch.2.013229}{Phys. Rev. Research {\bf 2}, 013229 (2020)}. 

\bibitem[S12]{Bernevig_App}
B. A. Bernevig, T. L. Hughes {\em Topological insulators and topological superconductors}, \href{https://collaborate.princeton.edu/en/publications/topological-insulators-and-topological-superconductors}{Princeton University Press, 2013}.

\bibitem[S13]{Spec_loc_5_App}
A. Cerjan and T. A. Loring, {\em Tutorial: Classifying Photonic Topology Using the Spectral Localizer and Numerical $K$-Theory}, \href{https://pubs.aip.org/aip/app/article/9/11/111102/3322376/Classifying-photonic-topology-using-the-spectral}{APL Photonics {\bfseries 9}, 111102 (2024)}

\bibitem[S14]{spec_loc_mobility_gap_App}
T. Stoiber, {\em A spectral localizer approach to strong topological invariants in the mobility gap regime}, \href{https://arxiv.org/abs/2410.22214}{ arXiv:2410.22214 (2024)}.

\bibitem[S15]{Fine_structure_App}
H. Liu, C. Fulga, E. J. Bergholtz, J. K. Asboth, {\em Topological fine structure of an energy band}, \href{https://arxiv.org/abs/2312.08436}{arXiv:2312.08436 (2023)}.

\bibitem[S16]{Chern_number_numerically_App}
T. Fukui, Y. Hatsugai, and H. Suzuki, {\em Chern Numbers in Discretized Brillouin Zone: Efficient Method of Computing (Spin) Hall Conductances}, \href{https://journals.jps.jp/doi/10.1143/JPSJ.74.1674}{
J. Phys. Soc. Jpn. {\bfseries 74}, pp. 1674-1677 (2005)}

\end{thebibliography}

\end{document}